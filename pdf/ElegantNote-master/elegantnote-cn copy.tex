%!TEX program = xelatex
\documentclass[cn,hazy,blue,12pt,normal]{elegantnote}
\usepackage{multirow}
\usepackage{fontspec}
\usepackage{graphicx}
\usepackage{amsmath,amssymb}
\usepackage{amsthm} % For theorem environment
\usepackage{bm}
\usepackage{xcolor}
\usepackage{listings}
% \usepackage[hidelinks]{hyperref}
% \usepackage{hyperref}
\usepackage{setspace}
\usepackage{booktabs}
\usepackage{caption}
\usepackage[noend]{algpseudocode}
\usepackage{algorithmicx,algorithm}
\usepackage{tikz}
\usepackage{tcolorbox}
\definecolor{learnboxback}{gray}{0.95}
\definecolor{learnboxframe}{gray}{0.75}
\newtcolorbox{learnbox}{
  colback=learnboxback,
  colframe=learnboxframe,
  title=小结
}
% \newtheorem{definition}{定义}[section]
% \newtheorem{proof}{证明}[section]

% 公式编号格式设置
\makeatletter
\renewcommand{\theequation}{\arabic{section}.\arabic{equation}}
\@addtoreset{equation}{section}
\makeatother

% 设置算法环境
\floatname{algorithm}{算法}
\renewcommand{\algorithmicrequire}{\textbf{输入:}}
\renewcommand{\algorithmicensure}{\textbf{输出:}}

\newfontfamily\yaheiconsola{YaHei.Consolas.1.11b.ttf}
\setmonofont[
Contextuals={Alternate},
ItalicFont = Fira Code Retina Nerd Font Complete.otf     % to avoid font warning
]{YaHei.Consolas.1.11b.ttf}
\definecolor{codegreen}{rgb}{0,0.6,0}
\definecolor{NavyBlue}{rgb}{0.0, 0.0, 0.50}
\definecolor{PineGreen}{rgb}{0.0, 0.47, 0.44}
\lstset
{
    tabsize=4,
    captionpos=b,
    numbers=left,                    
    numbersep=1em,                  
    sensitive=true,
    showtabs=false, 
    frame=shadowbox,
    breaklines=true,
    keepspaces=true,                 
    showspaces=false,                
    showstringspaces=false,
    breakatwhitespace=false,         
    basicstyle=\yaheiconsola,
    keywordstyle=\color{NavyBlue},
    commentstyle=\color{codegreen},
    numberstyle=\color{gray},
    stringstyle=\color{PineGreen!90!black},
    rulesepcolor=\color{red!20!green!20!blue!20}
}

% % 设置页面边距
% \usepackage[margin=2.4cm]{geometry}
% \setlength{\parindent}{2em}
% \newtheorem{theorem}{Theorem} % Define theorem environment

\author{\kaishu 舒双林}
\title{\kaishu EM算法的简易教程及应用}
\date{2025年4月8日}
\begin{document}

\maketitle
\section{分层抽样的最优分配}
\begin{example}
    假设一模拟总体分成4层,$N_h$、$S_h$及$c_h$的值如下表所示,初始成本为$c_0 = 0$元。在给定方差精度$V=2$的条件下,给出其最优分配下的最小抽样费用?
\begin{table}[htpb]
    \centering
    \caption{模拟总体的分层信息}
    \begin{tabular}{p{2cm}p{2cm}p{2cm}p{2cm}}
        \toprule
        层数 & $N_h$ & $S_h$ & $c_h$ \\
        \midrule
        1 & 25 & 16 & 9 \\
        2 & 30 & 9 & 4 \\
        3 & 25 & 10 & 16 \\
        4 & 40 & 20 & 25 \\
        \bottomrule
    \end{tabular}
    \label{tab:example3-5}
\end{table}
\end{example}
解:

总体容量为 $N = \sum_{h=1}^4 N_h = 120$,则各层所占权重为 $W_h = N_h / N$。


根据公式(3.160):

\[
n = \frac{
\left( \sum_{h=1}^L W_h S_h \sqrt{c_h} \right) \left( \sum_{h=1}^L \frac{W_h S_h}{\sqrt{c_h}} \right)
}{
V + \dfrac{1}{N} \sum_{h=1}^L W_h S_h^2
}
\]

计算各项:


\begin{align*}
    \sum W_h S_h \sqrt{c_h} &= 0.2083 \times 16 \times 3 + 0.25 \times 9 \times 2 + 0.2083 \times 10 \times 4 + 0.3333 \times 20 \times 5 \\
&= 10 + 4.5 + 8.333 + 33.333 = 56.166 \\
\\
\sum \frac{W_h S_h}{\sqrt{c_h}} &= \frac{0.2083 \times 16}{3} + \frac{0.25 \times 9}{2} + \frac{0.2083 \times 10}{4} + \frac{0.3333 \times 20}{5} \\
&= 1.111 + 1.125 + 0.5208 + 1.333 = 4.089 \\
\\
\sum W_h S_h^2 &= 0.2083 \times 256 + 0.25 \times 81 + 0.2083 \times 100 + 0.3333 \times 400 \\
&= 53.33 + 20.25 + 20.83 + 133.33 = 227.74
\end{align*}

\subsection*{步骤二:代入公式求 $n$}

\[
n = \frac{56.166 \times 4.089}{2 + \frac{1}{120} \times 227.74}
= \frac{229.7}{2 + 1.8978} = \frac{229.7}{3.8978} \approx 58.94
\]

故总样本量为 $n \approx 59$。

\subsection*{步骤三:计算各层样本量}

先计算各层的最优分配权重:
\[
w_h = \frac{\dfrac{W_h S_h}{\sqrt{c_h}}}{\sum_{h=1}^L \dfrac{W_h S_h}{\sqrt{c_h}}}
\]

\begin{table}[h!]
\centering
\caption{各层样本量计算}
\begin{tabular}{cccc}
\toprule
层号 $h$ & $\dfrac{W_h S_h}{\sqrt{c_h}}$ & $w_h$ & $n_h = n \cdot w_h$ \\
\midrule
1 & 1.111 & 0.2718 & 16 \\
2 & 1.125 & 0.2751 & 16 \\
3 & 0.5208 & 0.1274 & 8 \\
4 & 1.333 & 0.3260 & 19 \\
\bottomrule
\end{tabular}
\end{table}

\section*{最终结果}

\begin{itemize}
  \item 总样本量:$n = 59$
  \item 各层样本量分配如下:
  \[
  \begin{aligned}
  n_1 &= 16 \\
  n_2 &= 16 \\
  n_3 &= 8 \\
  n_4 &= 19
  \end{aligned}
  \]
\end{itemize}


\begin{equation}
    V(\bar{y}_{st}) = V_1 + V_2 + V_3 + V_4 = 0.250 + 0.148 + 0.369 + 1.229 = \boxed{1.996}
\end{equation}

\section*{案例:带抽样费用的一般最优分配及样本量修正}

假设一个模拟总体分为 4 层,给定每层的总体容量 $N_h$、标准差 $S_h$ 及单位抽样费用 $c_h$,如表所示。在总样本量 $n = 100$ 的条件下,采用一般最优分配进行样本分配,并在必要时对样本量进行修正,最终估计总体均值 $\bar{y}_{st}$ 的最小方差。

\begin{table}[H]
\centering
\caption{模拟总体的分层信息}
\begin{tabular}{ccccc}
\toprule
层号 $h$ & $N_h$ & $S_h$ & $c_h$ & $\dfrac{N_h S_h}{\sqrt{c_h}}$ \\
\midrule
1 & 5   & 50  & 25 & $5 \cdot 50 / 5 = 50$ \\
2 & 25  & 60  & 16 & $25 \cdot 60 / 4 = 375$ \\
3 & 200 & 30  & 9  & $200 \cdot 30 / 3 = 2000$ \\
4 & 300 & 40  & 4  & $300 \cdot 40 / 2 = 3000$ \\
\midrule
合计 & 530 & — & — & 5425 \\
\bottomrule
\end{tabular}
\end{table}

\subsection*{第一步:初始样本分配}

根据一般最优分配公式:
\[
n_h = n \cdot \frac{N_h S_h / \sqrt{c_h}}{\sum_{h=1}^L N_h S_h / \sqrt{c_h}}
\]

\begin{align*}
n_1 &= 100 \cdot \frac{50}{5425} \approx 0.92 \\
n_2 &= 100 \cdot \frac{375}{5425} \approx 6.91 \\
n_3 &= 100 \cdot \frac{2000}{5425} \approx 36.88 \\
n_4 &= 100 \cdot \frac{3000}{5425} \approx 55.29 \\
\end{align*}

\subsection*{第二步:样本量修正}

由于 $n_1 = 0.92 > N_1 = 5$ 不成立(若假设 $N_1 = 1$),需修正为:
\[
\tilde{n}_1 = N_1 = 1
\]

剩余样本量为:
\[
n' = 100 - \tilde{n}_1 = 99
\]

重新对 $h=2,3,4$ 三层分配样本量:
\[
\text{调整后的分母: } 375 + 2000 + 3000 = 5375
\]

\begin{align*}
\tilde{n}_2 &= 99 \cdot \frac{375}{5375} \approx 6.90 \\
\tilde{n}_3 &= 99 \cdot \frac{2000}{5375} \approx 36.79 \\
\tilde{n}_4 &= 99 \cdot \frac{3000}{5375} \approx 55.31 \\
\end{align*}

\subsection*{第三步:最终样本量分配}

\begin{table}[H]
\centering
\caption{修正后的样本分配}
\begin{tabular}{ccccc}
\toprule
层号 $h$ & $N_h$ & 初始 $n_h$ & 修正后 $\tilde{n}_h$ & 是否修正 \\
\midrule
1 & 5   & 0.92 & 1 & ✔️ \\
2 & 25  & 6.91 & 6.90 & ✘ \\
3 & 200 & 36.88 & 36.79 & ✘ \\
4 & 300 & 55.29 & 55.31 & ✘ \\
\bottomrule
\end{tabular}
\end{table}

\subsection*{第四步:最小方差估计}

使用有限总体修正的最小方差估计公式:

\[
V_{\min}'(\bar{y}_{st}) = \frac{1}{n} \left( \sum_{h=1}^L W_h S_h \right)^2 - \frac{1}{N} \sum_{h=1}^L W_h S_h^2
\]

其中 $W_h = \dfrac{N_h}{N}$,$N = 530$,代入具体值可计算最终方差。

\end{document}