\documentclass[a4paper,12pt]{ctexart} %A4纸,小四号字体
\usepackage{multirow}
\usepackage{fontspec}
\usepackage{graphicx}
\usepackage{amsmath,amssymb}
\usepackage{amsthm} % For theorem environment
\usepackage{bm}
\usepackage{xcolor}
\usepackage{listings}
\usepackage{hyperref}
\usepackage{setspace}
\usepackage{booktabs}
\usepackage{caption}
\usepackage[noend]{algpseudocode}
\usepackage{algorithmicx,algorithm}
\usepackage{tikz}
% \usepackage{tcolorbox} % 用于学习小结框
% \newtcolorbox{learnbox}{colback=blue!5!white, colframe=blue!75!black, title=学习小结}
% \newtheorem{definition}{定义}[section]
% // Removed the redefinition of the proof environment as it conflicts with amsthm
% \newtheorem{theorem}{Theorem}[section] % Define theorem environment

% 设置算法环境
\floatname{algorithm}{算法}
\renewcommand{\algorithmicrequire}{\textbf{输入:}}
\renewcommand{\algorithmicensure}{\textbf{输出:}}

\newfontfamily\yaheiconsola{YaHei.Consolas.1.11b.ttf}
\setmonofont[
Contextuals={Alternate},
ItalicFont = Fira Code Retina Nerd Font Complete.otf     % to avoid font warning
]{YaHei.Consolas.1.11b.ttf}
\definecolor{codegreen}{rgb}{0,0.6,0}
\definecolor{NavyBlue}{rgb}{0.0, 0.0, 0.50}
\definecolor{PineGreen}{rgb}{0.0, 0.47, 0.44}
\lstset
{
    tabsize=4,
    captionpos=b,
    numbers=left,                    
    numbersep=1em,                  
    sensitive=true,
    showtabs=false, 
    frame=shadowbox,
    breaklines=true,
    keepspaces=true,                 
    showspaces=false,                
    showstringspaces=false,
    breakatwhitespace=false,         
    basicstyle=\yaheiconsola,
    keywordstyle=\color{NavyBlue},
    commentstyle=\color{codegreen},
    numberstyle=\color{gray},
    stringstyle=\color{PineGreen!90!black},
    rulesepcolor=\color{red!20!green!20!blue!20}
}

% 设置页面边距
\usepackage[margin=2.5cm]{geometry}
\setlength{\parindent}{2em}

\title{EM算法}
\author{舒双林}
\date{\today}

\begin{document}
\maketitle

\tableofcontents
\newpage

\section{EM算法的理论基础}

\subsection{EM算法的引入}

EM(Expectation Maximization, 期望最大)算法是一种从不完全数据或含有隐变量的数据中估计参数的方法,由Dempster等人于1977年总结提出。EM算法的基本思想是:在每次迭代中,分两步进行,第一步是求期望(E步),即求隐变量的期望,第二步是求极大(M步),即求参数的极大似然估计。

一般地,用$Y$表示观测随机变量的数据,$Z$表示隐变量的数据。$Y$和$Z$组合一起称为完全数据,而观测数据$Y$称为不完全数据。假设给定观测数据$Y$,其概率分布是$P(Y|\theta)$,其中$\theta$是需要估计的模型参数,那么不完全数据$Y$的似然函数为$P(Y|\theta)$,对数似然函数$L(\theta)=\log P(Y|\theta)$;假设$Y$和$Z$的联合概率分布是$P(Y,Z|\theta)$,那么完全数据$Y$和$Z$的对数似然函数为$L(\theta)=\log P(Y,Z|\theta)$。

\subsection{EM算法的数学推导}

对于含有隐变量$Z$的概率模型,直接极大化对数似然函数$L(\theta)=\log P(Y|\theta)$是困难的,因为其含有未观测数据且包含和(或积分)的对数。

假设在第$i$次迭代后$\theta$的估计值是$\theta^{(i)}$。考虑新估计值$\theta$能否使$L(\theta)$增加,对两者作差:
\begin{equation}
L(\theta) - L(\theta^{(i)}) = \log \left( \sum_{Z} P(Y|Z,\theta)P(Z|\theta) \right) - \log P(Y|\theta^{(i)})
\end{equation}

利用Jensen不等式,得到其下界:
\begin{align}
L(\theta) - L(\theta^{(i)}) &= \log \left( \sum_{Z} P(Z|Y,\theta^{(i)}) \frac{P(Y|Z,\theta)P(Z|\theta)}{P(Z|Y,\theta^{(i)})} \right) - \log P(Y|\theta^{(i)}) \\
&\geq \sum_{Z} P(Z|Y,\theta^{(i)}) \log \frac{P(Y|Z,\theta)P(Z|\theta)}{P(Z|Y,\theta^{(i)})} - \log P(Y|\theta^{(i)})
\end{align}

定义辅助函数$Q(\theta,\theta^{(i)})$:
\begin{equation}
Q(\theta,\theta^{(i)}) = \sum_{Z} P(Z|Y,\theta^{(i)}) \log P(Y,Z|\theta)
\end{equation}

可以证明,最大化$Q(\theta,\theta^{(i)})$可以保证$L(\theta)$不减。因此,EM算法的迭代公式为:
\begin{equation}
\theta^{(i+1)} = \arg\max_{\theta} Q(\theta,\theta^{(i)})
\end{equation}

这等价于:
\begin{enumerate}
    \item E步:计算$Q(\theta,\theta^{(i)})$
    \item M步:求解$\theta^{(i+1)} = \arg\max_{\theta} Q(\theta,\theta^{(i)})$
\end{enumerate}

EM算法通过不断求解下界的极大化来逼近求解对数似然函数极大化,但不能保证找到全局最优值。

\begin{figure}[htbp]
    \centering
    \includegraphics[width=0.6\textwidth]{fig/EM_algorithm.png}
    \caption{EM算法示意图}
    \label{fig:em_algorithm}
\end{figure}

\subsection{EM算法流程}

\begin{center}
    \begin{minipage}{0.95\textwidth}
        \begin{algorithm}[H]
            \caption{EM算法} 
            \label{alg:em}
            {\bf 输入:} 观测变量数据$Y$、隐变量数据$Z$、联合分布$P(Y,Z|\theta)$、条件分布$P(Z|Y,\theta)$\\
            {\bf 过程:} 
            \begin{algorithmic}[1]
                \State 选择参数初值$\theta^{(0)}$、最大迭代次数$N$、迭代精度$\delta$,开始迭代
                \For{$i = 1$ to $N$}
                \State 令$\theta^{(i-1)}$为第$i-1$次迭代参数$\theta$的估计值
                \State E-step:计算在给定观测数据$Y$和当前参数$\theta^{(i-1)}$下$Z$的条件概率分布期望
                \begin{equation}
                        Q(\theta,\theta^{(i-1)}) = \sum_Z\log P(Y,Z|\theta)P(Z|Y,\theta^{(i-1)})
                \end{equation}
                \State M-step:极大化$Q(\theta,\theta^{(i-1)})$,确定第$i$次迭代的参数估计值$\theta^{(i)}$
                \begin{equation}
                    \theta^{(i)} = \arg \max_{\theta}Q(\theta,\theta^{(i-1)})
                \end{equation}
                \State 计算$\theta^{(i-1)}$与$\theta^{(i)}$的差值的二范数$\delta^{(i)}=||\theta^{(i)}-\theta^{(i-1)}||$
                \If{$\delta^{(i)}$ < $\delta$}
                \State 迭代结束,$\hat{\theta} = \theta^{(i)}$为参数的极大似然估计值
                \Else
                \State 继续迭代,$i = i + 1$,返回第3步
                \EndIf
                \EndFor
                \State 迭代结束,$\hat{\theta} = \theta^{(N)}$为参数的极大似然估计值
            \end{algorithmic}
            {\bf 输出:} 模型的参数估计值$\hat{\theta}$
        \end{algorithm}
    \end{minipage}
\end{center}

\subsection{关于EM算法的说明}
下面关于EM算法作几点说明:
\begin{enumerate}
    \item 参数的初值可以任意选择,但需注意EM算法对初值是敏感的。
    \item 迭代停止的条件可以是参数的变化小于一个给定的阈值,也可以是$Q$函数的增益小于一个给定的阈值,即
    \begin{equation}
    \|\theta^{(i+1)} - \theta^{(i)}\| < \delta_1 \quad \text{or} \quad \|Q(\theta^{(i+1)},\theta^{(i)}) - Q(\theta^{(i)},\theta^{(i)})\| < \delta_2
    \end{equation}
    \item M步求$Q(\theta,\theta^{(i)})$的极大化,得到$\theta^{(i+1)}$,完成一次迭代$\theta^{(i)} \to \theta^{(i+1)}$。后续将给出定理保证EM算法的收敛性。
\end{enumerate}

\subsection{EM算法的收敛性}
EM算法的收敛性由以下定理保证:

\begin{theorem}[单调性]
设 $P(Y|\theta)$ 为观测数据的似然函数,$\theta^{(i)}(i=1,2,\cdots)$ 为EM算法得到的参数估计序列,$P(Y|\theta^{(i)})(i=1,2,\cdots)$ 为对应的似然函数序列,则 $P(Y|\theta^{(i)})$ 是单调递增的。
\begin{equation}
P(Y|\theta^{(i+1)}) \geq P(Y|\theta^{(i)})
\end{equation}
\end{theorem}

\begin{theorem}[收敛性]
设 $L(\theta) = \log P(Y|\theta)$ 为观测数据的对数似然函数,$\theta^{(i)} (i = 1, 2, \cdots)$ 为 EM 算法得到的参数估计序列,$L(\theta^{(i)}) (i = 1, 2, \cdots)$ 为对应的对数似然函数序列。  

(1)收敛性:
如果 $P(Y|\theta)$ 有上界,则 $L(\theta^{(i)}) = \log P(Y|\theta^{(i)})$ 收敛到某一值 $L^*$。  

(2)稳定点性质:
在函数 $Q(\theta, \theta')$ 与 $L(\theta)$ 满足一定条件下,由 EM 算法得到的参数估计序列 $\theta^{(i)}$ 的收敛值 $\theta^*$ 是 $L(\theta)$ 的稳定点。
\end{theorem}

\subsection{EM算法的应用场景}

EM算法广泛应用于统计学、机器学习和数据挖掘等领域,尤其在处理缺失数据和隐变量模型时表现出色。以下是一些常见的应用场景:
\begin{itemize}
    \item 高斯混合模型(GMM)
    \item 隐马尔可夫模型(HMM)
    \item 潜在语义分析(LSA)
    \item 概率主成分分析(PPCA)
\end{itemize}
            
EM算法的优势在于:
\begin{itemize}
    \item 能够从不完全数据中有效估计模型参数
    \item 实现简单,计算效率较高
    \item 理论基础良好,易于理解和分析
    \item 可与其他算法结合使用(如变分推断、MCMC等)
\end{itemize}

\end{document}